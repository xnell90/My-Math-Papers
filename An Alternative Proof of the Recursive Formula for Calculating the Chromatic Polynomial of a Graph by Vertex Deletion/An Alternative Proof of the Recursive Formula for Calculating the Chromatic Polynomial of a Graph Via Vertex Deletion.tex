\documentclass[11pt]{article}

\usepackage{amsmath,amsfonts,amssymb,amsthm}
\usepackage[margin=1in]{geometry}
\usepackage[colorlinks]{hyperref}

\newcommand{\C}{{\mathbb{C}}}
\newcommand{\F}{{\mathbb{F}}}
\newcommand{\R}{{\mathbb{R}}}
\newcommand{\Z}{{\mathbb{Z}}}

\newcommand{\eq}[1]{(\ref{eq:#1})}
\renewcommand{\sec}[1]{Section~\ref{sec:#1}}

\newtheorem{thm}{Theorem}
\newtheorem*{thm*}{Theorem}
\newtheorem{cor}{Corollary}
\newtheorem{lem}{Lemma}
\newtheorem{defn}{Definition}
\newtheorem{prop}{Proposition}
\newtheorem{rem}{Remark}

\begin{document}

%%%%%%%%%%%%%%%%%%%%%%%%%%%%%%%%%%%%%%%%%%%%%%%%%%%%%%%%%%%%%%%%%%%%%%%%%%%%%%

\title{An Alternative Proof of the Recursive Formula for Calculating the Chromatic Polynomial of a Graph by Vertex Deletion}

\author{Alwyn G. Lopez \\
alwyn333@hotmail.com}

\date{\today}
\maketitle

%%%%%%%%%%%%%%%%%%%%%%%%%%%%%%%%%%%%%%%%%%%%%%%%%%%%%%%%%%%%%%%%%%%%%%%%%%%%%%
\begin{abstract}
This paper demonstrates a a double counting proof on the Recursive Formula for Calculating the Chromatic Polynomial of a Graph by Vertex Deletion.
\end{abstract}

%%%%%%%%%%%%%%%%%%%%%%%%%%%%%%%%%%%%%%%%%%%%%%%%%%%%%%%%%%%%%%%%%%%%%%%%%%%%%%
\section{Double Counting Proof}
\label{sec:intro}

We will state Jin's main theorem in its original form, and then prove it by a double counting argument. For more details on the notation, see \cite{Jin04}.

\begin{thm*}
Let $G$ be simple graph with $p$ vertices. Let $u \in V(G)$ be such that $d(u) = p - k$, where $d(u)$ denotes the degree of $u$ and $1 \leq k \leq p - 1$. Let $V_{u}^* = \{v_1, v_2, v_3,\ldots,v_{k - 1}\} \subset V(G)$ be the set of all vertices in $V(G)$ such that each vertex in $V_{u}^*$ is not adjacent to $u$. Then we have the vertex-deleting formula for the chromatic polynomial of the graph $G$,

\begin{equation}
P(G,\lambda) = \lambda P(G_u, \lambda - 1) + \lambda\sum_{H \subseteq V_{u}^*} P(G_{\{u\} \cup H}, \lambda - 1)
\end{equation}where the summation is extended over all independent sets $H \subseteq V_{u}^*$ with $ 1 \leq |H| \leq k - 1$. Here $G_J$ denotes the graph obtained from $G$ by deleting all vertices in $J$.
\end{thm*}

\begin{proof}
The left hand side of equation (1) counts the number of proper colorings on the graph $G$ with $\lambda$ colors, so it suffices to show that the right hand side of equation (1) also counts the number of proper colorings on the graph $G$ with $\lambda$ colors. Let $\bar{H}$ be an arbitrary independent set that contains the vertex $u$. Since there are $\lambda$ colors, pick one color to color all vertices in $\bar{H}$, delete all vertices in $\bar{H}$ including its associated edges, and finally use the remaining $\lambda - 1$ colors to properly color the graph $G_{\bar{H}}$. Then there are $\lambda P(G_{\bar{H}}, \lambda - 1)$ ways to do this. Since $\bar{H}$ is an arbitrary independent set that contains $u$, then we deduce that 

\begin{equation}
P(G,\lambda) = \lambda\sum_{\bar{H}} P(G_{\bar{H}},\lambda) \nonumber
\end{equation} gives us the number of ways to properly color the graph $G$ with $\lambda$ colors. Observe that the above sum is summing over all independent sets that contain $u$. Additionally, we know that $\bar{H} = \{u\} \cup H$, where $H \subseteq V_{u}^*$, so the above sum can be simplified to equation (1).

\end{proof}

%%%%%%%%%%%%%%%%%%%%%%%%%%%%%%%%%%%%%%%%%%%%%%%%%%%%%%%%%%%%

\begin{thebibliography}{9}
\bibitem{Jin04}
Xu Jin, \emph{Recursive Formula for Calculating The Chromatic Polynomial of a Graph via Vertex Deletion}, Acta Mathematica Scientia, 577-582 (2004).
\end{thebibliography}

\end{document}